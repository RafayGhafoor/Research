\documentclass[]{article}
\usepackage{geometry}
\usepackage{xcolor}
\usepackage{amsmath}
\usepackage[square, numbers]{natbib}
\usepackage[some]{background}

\definecolor{titlepagecolor}{cmyk}{1,.60,0,.40}

\backgroundsetup{
scale=1.42,
angle=0,
opacity=2,
contents={\begin{tikzpicture}[remember picture,overlay]
 \path [fill=titlepagecolor] (current page.west)rectangle (current page.north east); 
 \draw [color=white, very thick] (5,0)--(5,0.5\paperheight);
\end{tikzpicture}}
}

\makeatletter                   
\def\printauthor{%                  
    {\large \@author}}          
\makeatother

\author{%
    \textbf{FAST NUCES}\\
    Department of Computer Science \\
    \textbf{\newline Authors:} \\
    \item[$\cdot$] Rafay Ghafoor 
    \item[$\cdot$] Humza Ali
    \item[$\cdot$] Subhan Ahmed 
    \item[$\cdot$] Umer Shehzad 
    \item[$\cdot$] Zia Shahid 
    }

\begin{document}

\begin{titlepage}
\BgThispage
\newgeometry{left=0.4cm,right=0.5cm,bottom=1cm}
\vspace*{0.3\textheight}
\noindent
\textcolor{white}{\huge\textbf{\textsf{Contribution of Research in Economic Development}}} \\ \\
\textcolor{white}{\LARGE\textbf{\textsf{(A Comparison between Pakistan, Iran and Germany)}}} \\
\vspace*{3cm}\par
\noindent
\begin{minipage}{0.30\linewidth}
    \begin{flushright}
        \printauthor
    \end{flushright}
\end{minipage} \hspace{18pt}
%
\begin{minipage}{0.02\linewidth}
    \rule{1pt}{300pt}
\end{minipage} \hspace{-10pt}
%
\begin{minipage}{0.63\linewidth}
\vspace{5pt}
    \begin{abstract} 
    Research prior to the development of an idea's raw model has enlightened the methodological ways, the strategies for the implementation at application level, and enhanced innovation. This article emphasizes on the impacts research has brought us in different fields which cumulatively contribute in the economic development of a country.  Research in various sectors i.e., banking, education, medicine are subjected to study and their influence directly or indirectly on economic development of a country are observed. A comparison between two under-developed countries (Pakistan, Iran) and a developed country (Germany) is done to have a side by side analysis of which strategies countries incorporate in order to maintain their economic stability and its growth with time. Strategies for how the under-developed countries can strengthen and preserve their stability economic-wise is devised through the study of heuristics which has proved beneficial for the developed countries.\\
    \textbf{Keywords:} research, strategy, development, influence
    \end{abstract}
\end{minipage}
\end{titlepage}
\restoregeometry
\newpage

\newpage
\tableofcontents
\newpage

\section{Introduction:}

The article investigates the preliminary aspects that take part in contributing the economy of a state. Several important fields including banking, atomic energy, higher education, agriculture are subjected to study by analyzing developed country (Germany), its strategies are speculated, narrowing down to keep it to the economics sphere, to draw postulates which ultimately helps in contributing to the stability of economy and its growth. It also sheds light on the strategies which were adopted by different states leading to their downfall economy-wise which were afterwards deprecated due to several reasons i.e., high resources consumption (initially) impeding state's growth and eradicating resources exponentially, The observed postulates are then deduced and conclusions are drawn; a criterion is developed which devises a strategy, incorporation of which will help the under-developed countries to maintain their economy as well as deploy different techniques for augmenting their respective state's growth.

\subsection{Research Question:}

A comparative analysis between two under-developed countries i.e., Pakistan and Iran and a developed country Germany is made and research's emphasis is investigated - how ``research`` influences the economic development of a country. Different strategies and perspectives of Germany are investigated in order to learn the techniques adopted to become mature economic wise and how they can be incorporated by under-developed countries to strengthen their growth and preserve their stability.

\subsection{Limitations:}

The strategies adopted by the developed countries have few limitations which are discussed at length in this research article. A basic gist of the postulates drawn from the investigation of the analysis that may not be entirely beneficial for their incorporation by under-developed countries is due to their reliance upon geographical regions, market growth, trade factors, the relationship with other countries. The list is by no means exhaustive but the adoptation of the devised strategies proposed by the comparative analysis ensure the growth or at least have relational correspondence with the economy of the state, leaving enough room for positive points that can be taken away from this article.

\section{Literature Review:}

\subsection{Nuclear power in Economics:}

Countries which have centrally strategized economies, the total percent that nuclear energy contributes to the total electricity is roughly estimated about 13 percent of the total. If we research on the total nuclear capacity in developing countries, in the year 2000, 50 GWe electrical energy providing approximately 7 percent of their entire energy. Moreover, if the constraints of building nuclear power plants, their expenditure and fulling the basic requirements timely are overcome by manufacturing the nuclear power plants in the developing countries, the corresponding rise of about 30 percent of their total energy could be expected. 10 years, minimum are to be expected for the construction of power plant - the lead time.

Nuclear technology is one of the most significant source which can be used for generating electricity, heat and the countries planners should research on assessing the different states and kinds of energy, evaluating the supply options in order to to improve their nuclear power planning, ultimately improving their economy \cite{Hj03}.

\subsection{Higher Education augmenting innovation and enhancing awareness:}

The human capital view that is broadly accepted emphasizes that higher education augments existing skills and knowledge, resulting in increased outcome but the new researchers say otherwise i.e., many other factors are to be taken into consideration: geography, sectors, skills, education mechanisms, network in the companies are significant factors.
Higher education will continue to play an important role in economic development, As we start working towards the new sustainable goals, we will need professionals across all sectors - doctors, teachers and engineers will be vital to our future success, and education is vital to producing those professionals \cite{Hig15}. 

A focus on technological capabilities emphasizes how innovation and improvement come about through learning and communicating. The creation of distributed process of capacity development and network enhancement is a task in which the state must work particularly strategically with private and intermediary organiations, as well as educations and training organiations, themselves. A focus on learning, capabilities and interaction enables the indentifications of loopholes that may exist within the organizations; related to their capabilities, or outwardly within the system, itself \cite{Gle15}.

\subsection{Banking and its correspondence with technology model:}

The internet banking is the new bleeding edge technology which has revolutionized the banking methods by introducing banking transactions and has invoked new strategic directions for investment in banking and technologies. It has enabled customers to conduct personal and commercial banking activites in an efficient manner allowing them to use the services of bank without leaving their local place - services available globally. Thus, saving them the money that would be spent on travelling and more significantly, their precious time spend to visit the required branch to issue transaction. It is estimated that the operational cost to the bank of performing a transaction on the internet is about 0.01 doller while the cost of doing the same transaction at a bank branch is estimated to be 1.07 dollar (Sarel and Marmorstein, 2003; Nath, Schrick and Parzinger,
2001) \cite{Hum14}.

\subsection{Sources:}

The data is collected by observing different articles from different journals with good research indexing and the relevant information is included in form citations. Moreover, the link between the existing research and current research is established, driving towards the findings and conclusion. The data gathered is the interpretation from the existing frameworks used in research articles from different journals. However, the major focus of these sources is to create a link between the topic and its impact on the economics, describing the factors playing out the most important role in establishing and preserving of the economic state of a country.

\section{Research Methodology:}

The study focuses on the comparison between the growth of developing countries and their relation with a developed country. A qualititative analysis of journal research papers were conducted in the fields of banking, nuclear energy, agriculture and higher education
to identify the factors that contribute in preserving the stability and growth of a state, economic-wise.

\subsection{Criterion for Article Selection: (Review)}

Several databases were searched and different research articles were extracted from Google Scholar, read and reviewed. The articles were found by searching certain keywords such as ''Higher education and its relation with innovation in developed countries``, ''Influence of E-Banking on a state's economy``, ''Nuclear energy compensating several energy utilizing facilities by becoming a reservoir for many sources``. Additional parameters were further added to deepen the search. Studies from other countries were excluded since this research aims to restrict its focus on two developing countries that are Pakistan and Iran with their comparison to a developed country i..e., Germany for the conservation of brevity.

\subsection{Quality Assessment:}

Several articles were studies and a criterion for the assessment of their quality was developed which is by looking at the relevance of 
the material of the article to our comparison - relation with the economic growth. An analysis and scanning by reviewing the abstract, research methodology and its conclusion was observed under strict compliance with its relation to the growth of economy. The articles failing to meet the criterion were excluded.

\section{Discussion:}

CERN is on the top when particle physics Laboratories are discussed. One of its best creation is Large Hadron Collider (LHC). LHC is the largest and highest-energy particle accelerator. It has also been credited with the discovery of the Higgs Boson. Pakistan has been one of the countries contributing with CERN from 1960s with has been of financial assist. Dr Abdul Salam was the main reason then for CERN to work with Pakistan. Since then Pakistan has signed a no. of contracts with CERN. In 1997, Ishfaq Ahmad, Former chairman of the Pakistan Atomic Energy Commission, signed a contract between PAEC and CERN for contribution of 1 million Swiss francs. In 2000, another agreement was signed which increased the amount from 1 million Swiss francs to 2 million Swiss francs. In 2006, It had been nearly a decade for PAEC and CERN to be working together and PAEC was awarded by CERN as its Best Suppliers Award. Recently, CERB signed an agreement with Pakistan worth of total contribution for LHC as 10 million USD. On June 23th, 2013, Steve Myers, the director for CERN’s Large Hadron Collider (LHC) particle accelerator, on his visit to Pakistan said that Pakistan is in a better position to achieve stronger links with CERN than most other countries. He also said “Countries that apply for associate member status must have scientific infrastructure that is compatible with what we have at CERN,” and he further added “Pakistan has the necessary infrastructure to benefit from associate membership.” He told that though the countries like Ukraine and Brazil had applied for associate membership of the science body, Pakistani physicists were more focused towards an improved CERN link. Local physicists accept that a more grounded affiliation with CERN will win Pakistan greater access to test information from the LHC and will permit nearby companies to make electronic hardware for the best research facility. Myers, in conjunction with Fernando Quevedo, the director of the Abdus Salam Universal Centre for Theoretical Physics (ICTP), visited different universities and research organiations in Islamabad on Sunday. The directors said that they were ‘very impressed’ by the research work being carried out in Pakistan. “I think I was very impressed by the different institutes we visited,” Quevedo said. “The institutes are investing in the right way in both fundamental physics and applied physics.” He said ICTP encompasses a uncommon relationship with Pakistan because of Salam, the centre’s founding father. Pakistan is one of the nations that take an interest in a wide extend of ICTP programs each year, counting a three-month relate program, Quevedo said. Myers was on a two-day visit to Pakistan to go to the opening of the 38th yearly International Nathiagali Summer College (INSC), a gathering of researchers and analysts from around the world held in Nathiagali each year. In February 2014, CERN sent over its Technical team (CERN Technical Team /CTT) under the supervision of CERN’s Director for Research and Scientific Computing Dr Sergio Bertolucci. This team came for a four day visit to Pakistan to evaluate it for the associate membership of CERN. They also held meeting with the then Prime Minister Nawaz Sharif and President Mamnoon Hussain. The CTT also assessed and evaluated the scientific activities being undertaken in Pakistan. The Team also visited universities, scientific organisations and industrial complexes of Pakistan. According to the then PAEC Director PR Shahid Riaz, Pakistan was given preference by CERN because Pakistan has a high contribution in terms of manpower. At that time, nearly more than 30 Pakistani scientists were working in the research organisation. These are one of the main reasons that Pakistan became the first associate member state of CERN among Asian countries and is waiting to become a member state. All this contribution of Pakistan has enabled trust of other countries on Pakistan and proved that the image that foreign media portrays is not true. This has also opened a lot of career opportunities for Pakistanis in all the labs associated to CERN. But their benefits are not aligned. They are mismatched. Pakistan wants energy and weapons while CERN main purpose lies in advancement in Physics. This difference needs to be resolved in order for both to work together \cite {Ano10}.

Be that as it may, with respects to Germany, it is one of those countries which lead to the establishment of CERN. Nobel laureate Werner Heisenberg talked to Germany within the forerunner 'Conseil Européen pour la Récherche Nucléaire', and stamped the CERN Convention for the advantage of Germany in 1953. From that point forward, German analysts have made critical commitments to most major CERN projects. Herwig Schopper was the most German Director-General from 1981 until 1988 and Rolf-Dieter Heuer was the moment inhabitant from Germany, with a term of office from 2009-2015. . In Germany, research in particle and relativistic heavy ion physics is carried out predominantly by university groups.  Other fundamental central centres are two Max Planck Organizing (Munich and Heidelberg) and two inquire about centres of the Helmholtz Affiliation (DESY in Hamburg and Zeuthen, and GSI in Darmstadt). German industry has made crucial commitments to the improvement of the LHC, for illustration, the amassing of 33 percent of the large superconducting dipole magnets \cite {Ano15}.

Commitment of Iran with CERN begun in early 20’s. On 5 July 2001. Dr Mostafa Moin, Iranian Serve for Science, Inquire about and Innovation and Teacher Luciano Maiani, CERN1 Director-General nowadays marked an assertion for the interest of Iranian colleges within the Laboratory's logical program. Agreeing to this understanding, one Iranian analyst and three understudies will come to CERN to take an interest within the CMS try, with Iranian industry contributing to the experiment's development. The Reminder moreover clears the way for conceivable encourage Iranian involvement with tests at CERN. Presently Iran is a non-member state and is looking to be an associate member state. At present Iran is making a recharged and solid exertion to fix its relations with CERN and its scientific community, based on an Worldwide Participation Ascension which was marked in 2001. Iran is additionally taking an interest to the Grid infrastructure; the IRAN-GRID CA is facilitated by IPM. Nearly each year students and instructors take part to the Summer Understudy and Educator programs at CERN \cite{Pfas10}.

Germany’s economic freedom is about 73.5, which makes its economy more comfortable in the Index of 2019. Her overall score was reduced by 0.7 points, with a decrease in financial and business freedom, which increases the integrity of the government. Germany is divided into 14 of 44 countries in the European region, with its highest rates above provincial and international regions.

Freedom of business and freedom of investment remain stable in Germany. Long-term competition and business growth are supported by the opening of a global market, the protection of property rights and well-managed ecosystems. The current political partnership agreement indicates that a gradual reduction in monetary policy can be expected by increasing public investment in infrastructure and digital technology, higher child care costs and the use of low income tax. Government use to help reached record standards \cite{Ano19}.

The early signs of German settlement was noted at around 800 A.D. by the Holy Roman Empire. Their power was soon decentralized was partitioned into different princes. 19th century was the era in which many libertarians were born who strived to unite Germany. Adolf Hitler, their remarkable leader, created a sense of nationalism making them standout in the economy league making Germany one of the most dominant countries, worldwide\cite{Ste09}.

In the 15 months, there are only 3 out of all the worksheets, which are working in the background and the same way. This is a great deal, for example, for a printer. In 2003, Germany is a very good way to get the best of 31 m. At 18,8 times the data in the new window is located in this area. In addition, it is also one of the most common pages in the main area of ​​the site and it is the same way in this article, because it is a source of information. When you have a look at the top of the list as a sphere, where you are, a number of data, a lot of information, and an intermediate network, you have a great idea of ​​it, as it is in a series, as well as in a series. At the top of the screen, you will have a 10-bit look at the top of the list, and look at the most common features of the application. Not only does it have to change the change. If you look at sustainability science as an interdisciplinarsci-ence, where natural, technical, social sciences and humanities ideally come together equally, then there is still a long way to go, both internationally as well as in Germany. Environmental sciences are still dominating everywhere, and they are very of-ten still strongly focused on their own fi  eld od studies, despite rapidly increasing interdisciplinary offers (cf. details further below). It is not only here where a clear change is visible. When the Federal Environment Agency (FEA) in 1977 firstly  published  a study guide for environment protection, post-treatment stra-tegies were still dominating: waste disposal, regeneration of polluted rivers, in short, end-of-pipe strategies were the focus of attention in education and research. The study opportuni-ties in this field increased quickly \cite{Ger09}.

\section{Findings:}

 The finding provides insightful directions and research gaps on higher studies, nuclear energy, banking and will be helpful to academics who are working or aim to work in the respective afore-mentioned areas in developing countries.
 
 
 
 International trade require countries and their economies to compete with each other. 
 The countries which are successfull economic-wise will always be in competition against other economics of different countries and have significant comparative advantages over other economies though to get specialization in a particular industry is a rare feat. This means the country's economy consists of various advantages which have their own pros and cons in global marketplace. Therefore, the education and training of a country's workers is one of the crucial factor that determines how the economy of a country will play out in the foreseeable future.
 
 Hypothesis 1: Education and training of a country's worker should be a must and a significant investment should be buried in this procedure to ensure the economic growth and establishment of a country.
 
 Miscellaneous decision that are related to other technological options when chosen to create additional dependencies that may have a correlation with eletricity generation chiefly comprises of evaluating the number of other choices and their costs in real application. Thus, theorizing and strategizing the former end, understanding the parameters that require the initial understanding of the introductory framework, related to the working of the system and its underlying base.
 However most of the time these costs do not fully reflect the broader impacts ("externalities") of this energy choice on the economy and society at large. To formulate their future energy and resource development policies governments therefore have to take them into account whether of an economic environmental health or social nature which may support or discourage the adoption of a particular technology.
 
 Hypothesis 2: Energy reservoirs should be established to ensure the growth in future and the preservation of economic environmental health or social nature, adopting new technologies and strategizing to preserve the energy. Moreover, investing it efficiently without losing it have more benefits ratio to the provided input.
 
Banking system play a crucial role in determining the financial system and the economy of a state. The stronger the banking system, the more surety and establishmed of a state regarding to their economy. Moreover, a balance could be created by obtaining funds, loans which could be easily accessible within the framework due to the stable growth in the former tenure, leading to a stable growth in foreseeable future. Since, the banks give specialized financial services, which minimizes the cost of obtaining information about both savings and borrowing opportunities. These financial services help to make the overall economy more efficient. 

Hypothesis 3: Banking system should be made as powerful as possible, utilizing the economy effectively since it promises the stable growth of a country.

\section{Conclusion:}

We argue that international education and development thinking about the relationship between education, technological innovation, production and development should be given heed and projects regarding higher education must be initated, to not only to improve literacy rate, awareness but to augment the research flow ultimately influencing the economics of a state. Similar to higher education, as discussed earlier nuclear power also contributes heavily as a source for the many form energies; It is expected to establish nuclear power plants to improve the state's economic health. Furthermore, the banking systems providing a unified interface for the people to interact with their money in the whole globe should be a motivation for the countries to establish a systematic mechanism that allows the throughout of money and its circulation in a pre-defined manner - gaining as much from the capital i.e., taxes etc.

\bibliography{myRef}
\bibliographystyle{newapa}
\end{document}
