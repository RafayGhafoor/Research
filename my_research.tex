\documentclass[]{article}
\usepackage[T1]{fontenc}
\usepackage[utf8]{inputenc}
\usepackage{authblk}
\usepackage{fancyhdr}
\usepackage{natbib}
\pagestyle{fancy}
\fancyhf{}
\rhead{Article}
\lhead{``Research``}
\rfoot{Page \thepage}

\title{A Comparative Analysis between Pakistan, Iran and Germany About "Research” Contribution in Economic Development and Its Indirect Impact on Establishing Society}
\date{\today}
\author[1]{Rafay Ghafoor}
\author[2]{Humza Ali}
\author[3]{Subhan Ahmed}
\author[4]{Zia Shahid}
\author[5]{Umer Shehzad}
\affil{Department of Computer Science, FAST NUCES}


\usepackage{titling}
\renewcommand\maketitlehooka{\null\mbox{}\vfill}
\renewcommand\maketitlehookd{\vfill\null}

\begin{document}

\begin{titlingpage}
\maketitle
\end{titlingpage}

\newpage

\newpage
\tableofcontents
\newpage


\section{Abstract}
\textbf{Keywords:}
\section{Introduction}
\section{Literature Review}
\section{Methodology}
\section{Findings}
\section{Conclusion}
\section{References}
\cite{Wel03} presents some methodology on how to perform tests of equivalence,and also the one-sided versions of these, which are known as non-inferiority ornon-superiority tests. On the other hand, if you want to know about receiveroperating characteristic (ROC) curves, a good source of information is \cite{Met78}.

\bibliography{myref}
\bibliographystyle{newapa}
\end{document}
